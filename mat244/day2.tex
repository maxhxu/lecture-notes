\section{Day 2: First Order Linear ODEs (Jan 07, 2026)}

Sometimes, there is no better way to describe a function than as a solution of a DE. For example, $e^{-x^2}$ has no elementary antiderivative.

\begin{definition}[First Order Linear DE]
    An equation of the form $y' + p(t)y = g(t)$ where $p$ and $g$ are some functions of $t$. It homogeneous if $g(t) = 0$ for all $t$.
\end{definition}

When solving for $A$ in $A'' = -9.8 - 0.17A'$, notice that $A$ is not present. Make the substitution $V = A'$, giving $V' = -9.8 - 0.17V$, which is a first order linear DE.

\begin{problem}
    Find the general solutions to each of $y' = ay$, and $y' = ay + b$.
\end{problem}

Take $y = ce^{at}$. See that for arbitrary $c$, $\frac{d}{dt} ce^{at} =  ace^{at} = ay$

Take $y = ce^{at} - \frac{b}{a}$.

\begin{problem}
    Find the general solutions to each of $y' = g(t)$, and $y' = f(t) y$.
\end{problem}

The general solution of $y = \int g(t) \, dt$ is acceptable.

The general solution is $c \Exp (\int f(t) \, dt)$

\begin{simplethm}
    The general solution of $y' = p(t)y + g(t)$ is
    \[
        y = e^{P(t)} \left( c + \int e^{-P(t)} g(t) \, dt \right)
    \]
    where $P(t) = \int_{t_0}^t p(s) \, ds$, and $c$ is an arbitrary constant.
\end{simplethm}

% Let $F(t)$ be a primitive of $f$. $F(t)(y - y')$

\subsection{Integrating Factors}

The product rule says that $f(t) = \frac{d}{dt} (\mu(t)y(t)) = \mu(t)y'(t) + \mu'(t)y(t)$. Dividing both sides by $\mu(t)$, see that this is a first order linear DE,
\[
y' + \frac{\mu'}{\mu}y = y' + p(t)y = g(t) = \frac{f}{\mu}
\]

Solve for $\mu$ in $\frac{\mu'}{\mu} = p(t)$. Finally, get $y(t) = \frac{\int f(t) \, dt + c}{\mu(t)} = \frac{\int g(t)\mu(t) \, dt}{\mu(t)}$.
