\section{Day 2: Arguments (Sept 5, 2025)}

In this class units do not correspond to weeks.

Arguments are made of statements, which are either true (T) or false (F). In this class, `sentences' are equivalent to statements, and all arguments here are deductive. Deductive arguments are certain, meaning the conclusion must follow.

\textbf{Validity} is about form and structure, and is a property unique to arguments. To show validity, suppose all premises are true and prove the conclusion follows.

An argument is \textbf{sound} if and only if it is both valid and all premises are true statements.

\subsection{Symbols for Sentiential Logic}
\begin{description}
    \item[Atomic Statements] capital letters P-Z
    \item[Logical Connectives] $\sim \text{ or } \neg, \to , \leftrightarrow, \wedge, \vee$
    \item[Organization] {(), []}
\end{description}

\subsection{Well Formed Formulae}
\subsubsection*{Formal Notation Rules:}
\begin{description}
    \item[Sentence Letters] P-Z by themselves
    \item[Unary Connectives] only applies to $\sim$ or $\neg$, you cannot have parentheses
    \item[Binary Connectives] must be in parentheses
\end{description}
\subsubsection*{Informal Notation Rules:}
\begin{enumerate}
\item Parentheses over some connectives
\item Since $\wedge$ and $\vee$ commute, you don't need parentheses for chains of them (rightmost one is the main connective)
\end{enumerate}

\begin{definition}[Consistent]
For a set of statements, there exists a truth value assignment such that all statements are true.
\end{definition}
\begin{definition}[Tautology]
A sentence that is always true.
\end{definition}
\begin{definition}[Contradiction]
A sentence that is always false.
\end{definition}
\noindent There exist \textit{contingent} statements, that are neither always true nor always false.
