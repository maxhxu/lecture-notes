\section{Day 1: (Sept 9 2025)}

\begin{remark}
I know why you're all here. I do follow the UofT subreddit - Alex Koo
\end{remark}

\noindent TLDR: First class of the school year! \textbf{Nothing is due until the third friday of September} (the first test). Otherwise quizzes due every Friday at midnight.

\begin{itemize}
\item Everything about this course is on \textbf{Quercus}, you can complete the entire course's content ahead of time if you want. 
\item \textbf{Piazza} is the discussion board, which supports fancy math notation.
\item \textbf{Crowdmark} is where you will be able to see your marked tests
    \item \textbf{Logic2010} is where all weekly quizzes will be assigned (apart from the first one which is on Quercus)
    \item \textbf{Mentimeter} is a way to stay engaged in class (?)
\end{itemize}

There are 4 in class tests, worth 55\% total. The final exam is worth 35\%. This means that 90\% of the mark are assessments taking place in class.\footnote{I think this is new, last yr wasn't like this} On tests, you will feel the time crunch, according to the Professor (Alex Koo). You can read through the syllabus to get a more detailed breakdown.

Generally, there is no class on Friday, because that's when tests happen. The lecture on Tuesday is for practice.

There are \textbf{readings}, but there is nothing in them that will jumpscare you on the test. In fact, there may be things in the readings that are completely useless.

The philosophy department runs a logic lab, with 5 TAs, with 20 hours of support per week (!) They're a mix of undergrad and grad students, and are extremely talented.

\begin{remark}
Logic2010 is the first time many of you will download an executable and run it yourself, and I know this may be a challenge, especially for you \textit{Mac users}. You're gonna be like: ``Wait, this isn't an app!'' - Prof Koo 2025
\end{remark}

You will need to register in the right section, but since our section is the default I doubt this will be an issue. \textbf{On Logic2010, you will need to submit each question individually.}
