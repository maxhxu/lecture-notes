\section{Day 4: Conditional Probability (Sept 15, 2025)}

From last class: choosing a subset in order is called a permutation, choosing a subset irrespective of the order is called a combination. You don't have to use R in this course, but if you wanted to there is some info \href{http://probability.ca/Rinfo.html}{here}.

\begin{problem}
    Suppose we flip 4 fair coins, what is $P(\text{exactly 2 heads})$?
\end{problem}

\noindent You could solve this by writing out the entire sample space. Or by computing $\frac{\binom{4}{2}}{2^4} = \frac{3}{8}$. In general, for flipping $n$ coins, the probability of getting exactly $k$ heads is
\[
\frac{\binom{n}{k}}{2^n} \text{for $0 \leq k \leq n$}
\]

\subsection{Conditional Probability}

We now receive some information that restricts the sample space of interest to some subset of the original sample space $S$. If $P$ was a discrete uniform distribution, $P$ on said subset is remains a discrete uniform distribution. 

\begin{definition}[Conditional Probability]
    If $A$ and $B$ are two events, where $P(B) > 0$, then the \textit{conditional probability} of $A$ given $B$ is written $P(A \mid B)$  represents the fraction of the times when $B$ occurs, in which $A$ also occurs. 
\[
P(A \mid B) = \frac{P(A \cap B)}{P(B)}
\]
\end{definition}

If $P(B) = 0$, then $P(A | B)$ is undefined. Assuming that an event with probability 0 occurring would lead to all sort of contradictions that I don't want to explore.

\begin{simplethm}[Conditional Multiplication Formula] \label{thm:condmultf}
    \[P(A \cap B) = P(A)P(B \mid A) = P(B)(A \mid B)\]
\end{simplethm}

Combining with law of total probability \ref{thm:lawoftotalprob} we can get a more useful version where we replace $P(A_i \cap B)$ according to \ref{thm:condmultf}, giving
\[
P(B) = \sum_i P(A_i)P(B \mid A_i)
\]
which is a lot more useful. Remember that $\bigcup_i A_i = S$. 

\begin{problem}[Challenge]
Roll $n$ fair six sided dice. What are the odds we get more than $0 \leq k \leq n$ 5s? 
\end{problem}

