\section{Day 5: Independence (Sep 17, 2025)}

When solving problems, solve them systematically, it is very easy to get tricked. English is already a complex language, and reading comprehension is even more difficult. Don't rely on analogies to familiar phenomena, when asked to obtain results do everything from the definition.

We start this class with some review of conditional probability. The main challenge in conditional probability questions is finding the correct partition for $S$. 

Recall the conditional multiplication formula \ref{thm:condmultf}. We are given $P(A)$, $P(B)$, and one of $P(A \mid B)$ or $P(B \mid A)$. To solve for the other, we can use the following (simply divide by $P(B)$)

\begin{simplethm}[Bayes Theorem]
\[
    P(A \mid B) = \frac{P(A)}{P(B)} P(B \mid A)
\]
\end{simplethm}

\subsection{Independence}
If $A$ and $B$ are any to events, to say that they are independent is that knowing that one happens does not effect the probability of the other. We could define independence as $P(A \mid B) = P(A)$ and $P(B \mid A) = P(B)$, but they are undefined for $P(A), P(B) = 0$. To try to see how we can deal with this drawback, we look into the definition of conditional probability, where we find that we are dividing by $P(B)$.
\begin{align*}
    P(A \mid B) = \frac{P(A \cup B)}{P(B)} &= P(A) \\
    P(A \cup B) &= P(A)P(B)
\end{align*}
which is a much better definition, as it's more symmetric, easier to work with, and defined for more events.
\begin{definition}[Independence]
    We say 2 events $A, B \subseteq S$ are independent if 
    \[
    P(A \cup B) = P(A)P(B)
    \]
\end{definition}

So far, independence is defined pairwise. What about triplets, or more? We cover that next class.
