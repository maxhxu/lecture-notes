\documentclass[9pt]{article}
\usepackage{aten, booktabs, tabularx, array}

\usepackage{enumitem}
\setlist[description]{leftmargin=1em, labelindent=1em}

\setcounter{tocdepth}{1}

\newcommand{\documenttitle}{MAT237 Optimization Cheat Sheet}
\newcommand{\authorname}{Max Xu}

\setkeys{Gin}{width=0.95\textwidth}

\begin{document}

\title{\documenttitle}
\rhead{\textbf{\small \documenttitle}}
\lhead{\textbf{\small \authorname}}
\author{\authorname}

\section{Optimization Workflow}

Suppose you want to find maxima/minima of some $f : \mathbb{R}^n \to \mathbb{R}$, on the set $A \subseteq \mathbb{R}^n$. This is equivalent to finding maxima/minima of $f \big|_A : A \to \mathbb{R}$.

\begin{enumerate}[(i)]
    \item Determine whether global extrema must exist.

        Sufficient conditions for global extrema include: 
        \begin{itemize}
            \item EVT \autoref{thm:evt}: requires A to be compact, $f\big|_A$ to be continuous
        \end{itemize}
    \item Identify critical points on the interior.

        Properties of critical points $a$ on the interior:
        \begin{itemize}
            \item Local EVT \autoref{thm:localevt}: $\nabla f(a) = 0$ or $\nabla f(a)$ DNE
        \end{itemize}
        
    \item Check the boundary for extrema.

        There are 2 main ways of checking the boundary:
        \begin{description}
            \item[Parameterization] Find some $g : B \to A$, where $B \subseteq \mathbb{R}^m$ with $m < n$, such that $\mathrm{im}(g) = A$. This lets you solve the lower ($m$) dimensional optimization problem of $f \circ g$.
            \item[Lagrange] Requires that $f$ is $C^1$. Find some $g : \mathbb{R}^n \to \mathbb{R}$, such that $\partial A = g^{-1}(\{ c \})$ for some $c \in \mathbb{R}$, with $\nabla g(p) \ne 0$ for all $p \in \partial A$.

                Then for each local extrema $a$ on $S$, there exists some $\lambda \in \mathbb{R}$ such that $\nabla f(a) = \lambda \nabla g(a)$.
        \end{description}
    \item Check all candidates

        Interior and boundary points are only candidates. Global maximums are local maximums, and the same applies to the subsets case.
\end{enumerate}

\section{Relevant Theorems}
\begin{simplethm}[EVT] \label{thm:evt}
    If $A \subseteq \mathbb{R}^n$ is a non-empty compact set, and $f : A \to \mathbb{R}$ is continuous, then $f$ attains maximum and minimum values at points of $A$.
\end{simplethm}

\begin{simplethm}[Local EVT] \label{thm:localevt}
    Let $A \subseteq \mathbb{R}^n$, and let $f : A \to \mathbb{R}$ be a real-valued function. If $a$ is an interior point of $A$, and $f$ has a local extremum at $a$, then $\nabla f(a) = 0$ or $\nabla f(a)$ DNE.
\end{simplethm}

\begin{simplethm}
    Let $U \subseteq \mathbb{R}^n$ be an open set, and $f : U \to \mathbb{R}$ be $C^1$. Suppose $S = f^{-1}(\{ 0 \})$ is non-empty, and fix $p \in S$. If $\nabla f(p) \ne 0$, then $S$ is a $(n-1)$ dimensional smooth manifold at $p$. A vector $v \in \mathbb{R}^n$ is a tangent vector of $p$ if and only if $\nabla f(p) \cdot v = 0$.
\end{simplethm}


\end{document}
