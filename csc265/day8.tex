\section{Day 8: Average/Expected-Case Analysis (Sep 24, 2025)}

Probability knowledge that you need: basic probability, expectation, linearity of expectation, Markov's inequality. If something is not in CLRS, it will be given to you, so no surprises (hopefully).

Let \textsc{A} be an algorithm. Let $S_n$ be a sample space containing inputs of size $n$ for $A$. Let $t_n(x)$ be the number of steps $A$ takes on input $x \in S_n$. Suppose a probability distribution $\text{Pr}$ is given over $S_n$. Typically, this is given by the problem or assumed (e.g. uniform).

\begin{definition}[Average Case Step Complexity]
\[
    T_{AV}(n) = \mathbb{E}[t_n] = \sum_{x \in S_n} \text{Pr}(x) t_n(x) = \sum_{\text{$i$ in range of $t_n$}} i \, \text{Pr}(t_n = i)
\]
\end{definition}

$\text{Pr}(t_n = i) = \text{Pr}(t_n^{-1}(i))$. where $t_n^{-1}$ is the inverse image, the probability of the set that gets mapped to $i$ by $t_n$. Now, to compute $T_{AV}(n)$, we go through the following steps:
\begin{enumerate}
\item Define a sample space $S_n$.\\
    Choosing an easy sample space will make your life easier.
\item Given $\text{Pr}$ distribution over $S_n$.
\item Define $t_n$ and any other random variables.\\
Choose something reasonable.
    \item Decide which $T_{AV}(n)$ formula you want to use. \\
    This step relies heavily on intuition and experience, but typically you try to partition the sample space using another random variable (like an indicator).
\end{enumerate}

If the step count is based on the number of comparisons, then only the relative order among elements does matter. An easy $S_n$ can be the permutations of the set $\{ 1, \cdots, n \}$, and a simple $\text{Pr}$ is the uniform distribution.\footnote{will we ever encounter a continuous distribution in cs?} 

There exist `nicer' ways to find $T_{AV}(n)$ for some algorithms that don't require solving recurrences, but they might be harder to find. 

Let $t_n(x, \sigma)$ be the number of steps taken by a randomized algorithm \textsc{A} on input $x$ with a sequence of random choices $\sigma$. It can be very very hard to define $\sigma$, especially if future choices depend on past choices, so generally we hand-wave the specifics.
\begin{definition}[Expected Case Step Complexity]
    The expected step complexity of \textsc{A} on input $x$ is
    \[
    \underset{\sigma}{\mathbb{E}} [t_n(x, \sigma)] = \sum_ \sigma \text{Pr}(\sigma) t_n(x, \sigma)
    \]
\end{definition}
\begin{definition}[Worst Case Step Complexity]
\[
    T_{EX} = \underset{x \in S_n}{\text{max}} \{  \mathbb{E}_ \sigma [t_n(x, \sigma)] \}
\]
\end{definition}
