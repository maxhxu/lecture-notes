\section{Day 5: Relational Algebra III (Jan 21, 2026)}

Notice that the cross-product of a relation with itself currently creates ambiguity in references to columns. We introduce the rename operator to resolve this ambiguity.
\begin{description}
    \item[Rename] Denoted by $\rho$, it renames a relation, subject to scoping constrants. The renamed relation is only available in ancestor nodes in the relational expression tree.
\end{description}

For example, $\rho_{B(foo, bar)} A$, where $A$ is a relation of form $A(attr1, attr2)$, is equivalent to renaming the relation $B$, the column $attr1$ with $foo$, and the column $attr2$ with $bar$.

The remainder of class was spent working on a worksheet.
