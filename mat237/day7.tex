\section{Day 7: Interior, Boundary, and Closure (Sep 16, 2025)}

Let $A \subseteq \mathbb{R}^n$.

\begin{definition}[Interior Point]
    $p \in \mathbb{R}^n$ is an interior point f $A$  if there exists an $\epsilon > 0$ such that $B_ \epsilon (p) \subseteq A$. 
\end{definition}

\begin{definition}[Interior]
The interior denoted $A^o$ is the set of all interior points of $A$. 
\end{definition}

\begin{definition}[Boundary Point]
    $p \in \mathbb{R}^n$ is a boundary point if for every $\epsilon > 0$, $B_ \epsilon(p) \cap A$ and $B_ \epsilon \cap A^c$ are non-empty.
\end{definition}

The boundary point can be isolated, there may exist $\epsilon$-balls at $p$ such that $p$ is the only point in said ball.

\begin{definition}[Topological Boundary]
The topological boundary of $A$, written $\partial A$.
\end{definition}

\begin{definition}[Limit Point]
    $p \in \mathbb{R}^n$ is a limit point of $A$ if for every $\epsilon > 0$, $B_ \epsilon \setminus \{ p \}$ contains points in $A$. The set of all limit points is written $A^*$.
\end{definition}

A set $B$ containing points in $A$ is the same as saying that $A \cap B \ne \emptyset$. Also to trip you up, sometimes questions will need you negate the definition, so do that if it seems unintuitive to explain.

\begin{definition}[Closure]
    The closure of $A$ is written $\overline{A}$, defined as $\overline{A} = A^* \cup A$.
\end{definition}

\begin{simplethm}
Every interior point of $A$ is a limit point of $A$. That is $A^o \subseteq A^*$.
\end{simplethm}
\begin{proof}
    Let $p$ be an interior point of $A$. By definition, exist $\epsilon > 0$, such that $B_ \epsilon (p) \subseteq A$. Let $\epsilon' > 0$ be arbitrary. If $\epsilon' > \epsilon$, note that $B_ \epsilon(p) \subseteq B_{\epsilon'}(p)$e can take any point in the open ball $B_ {\epsilon}(p) \setminus \{ p \}$, which is a subset of $A$ following from the definition. If $\epsilon' \leq \epsilon$, we can take any point in $B_{\epsilon'}(p) \setminus \{ p \} \subseteq B_ \epsilon(p) \setminus \{ p \} \subseteq A$.
\end{proof}

Note that this proof relies on the fact for $\epsilon > 0$, there exist points in $B_ \epsilon (p) \setminus \{ p \}$ that are not $p$. This is not true from some topologies. 

\begin{simplethm}
\begin{align*}
    A^o \subseteq A \subseteq \overline{A} &\\ A^o \cap \partial A = \emptyset &\\ \overline{A} = A^o \cup \partial A &\\ \partial A = \overline{A} \setminus A^o
\end{align*}
\end{simplethm}

You should be able to verify this by wrangling definitions.
