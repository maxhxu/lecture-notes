\section{Day 2: Stuff (Sept 9, 2025)}

We want to differentiate and integrate functions $A \to B$, where $A \subseteq \mathbb{R}^m$, $B \subseteq \mathbb{R}^n$. Today we study the case where $m = 1$, so functions $A \subseteq \mathbb{R} \to \mathbb{R}^n$. Today, will mainly look at functions $\mathbb{R} \to \mathbb{R}^n$, as having a single parameter makes them much easier to work with.

Recall that distance is a scalar quantity, while velocity is a vector, meaning it has both magnitude and direction. Have $f: \mathbb{R} \to \mathbb{R}^n$ model some particle's position, and $||\cdot||$ be the euclidean norm. The average speed\footnote{TODO: I don't believe we are concerned with the `actual' speed over time in this course, since that would involve finding the arc length and even more trouble. I'm unsure about this, for now I will assume it means `magnitude of the displacement vector'.} over the time interval $t_1$ and $t_2$ ($t_1 < t_2$) is given by
\[
\frac{||f(t_2) - f(t_1)||}{t_2-t_1}
\]
The instantaneous speed at time $t$ is given by
\[
\lim_{ h \to 0 } \left|\left|\frac{f(t + h) - f(t)}{h} \right|\right|
\]
The average velocity between $t_1$ and $t_2$ is given by
\[
\frac{f(t_2) - f(t_1)}{t_2 - t_1}
\]
and the instantaneous velocity at $t$ by
\[
\lim_{ h \to 0 } \frac{f(t + h) - f(t)}{h}
\]
\begin{theorem}[Absolute Homogeneity of Euclidean Norm] \label{thm:euclidabshog}
The euclidean norm has the \textit{absolute homogeneity} property. \\
    For all $v = \begin{pmatrix}
v_1 \\ \vdots \\ v_n
\end{pmatrix} \in \mathbb{R}^n$, a scalar $\lambda$
\[
|| \lambda v || = |\lambda| \, ||v||
\]
\end{theorem}
\ref{thm:euclidabshog} and various properties of norms were not mentioned in class.\footnote{I included this for completeness because people didn't believe that $\frac{||\gamma(6 + h) - \gamma(6)||}{|h|}$ and $\left|\left| \frac{\gamma(6 + h) - \gamma(6)}{h} \right|\right|$ were the same quantity.}
\begin{proof}

\begin{align*}
    ||\lambda v|| &= \left|\left|\begin{pmatrix}
\lambda v_1 \\ \vdots \\ \lambda v_n
\end{pmatrix} \right|\right| \\
    &= \sqrt{\lambda^2 {v_1}^2 + \dots + \lambda^2 {v_n}^2} \\
    &= \sqrt{\lambda^2 ({v_1}^2 + \dots + {v_n}^2)} \\
    &= |\lambda| \sqrt{{v_1}^2 + \dots + {v_n}^2)} \\
    &= |\lambda| \, ||v||
\end{align*}

\end{proof}
