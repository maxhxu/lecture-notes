\section{Day 8: Sequences (Sep 18, 2025)}

\begin{definition}[Sequence]
A sequence in in $\mathbb{R}^n$ is a function with domain $\{ k \in \mathbb{Z} : k > k_0 \}$ for some fixed $k_0 \in \mathbb{Z}$ and codomain $\mathbb{R}^n$. 
\end{definition}

\begin{definition}[Subsequence]
    Let $x : \mathbb{N}^+ \to \mathbb{R}^n$ be a sequence and $m : \mathbb{N}^+ \to \mathbb{N}^+$ be a strictly increasing function. The sequences $\{ x(m(k)) \}_{k=1}^\infty$ is a subsequence of the sequences $\{ x(k) \}_{k=1}^\infty$.
\end{definition}

\begin{definition}[Convergence]
    We fix a $p \in \mathbb{R}^n$. A sequence $\{ x(k) \}_k$ in $\mathbb{R}^n$ converges to $p$ if for every $\epsilon > 0$, there exists a $K \in \mathbb{N}$ such that for every $k \in \mathbb{N}$, if $k \geq K$ then  $||x(k) - p|| < \epsilon$.
\end{definition}

\begin{simplethm}
    A sequence $\{ x(k) \}$ converges to a point $p$ if and only if for every $\epsilon > 0$, the set of indices $\{ k \in \mathbb{N}^+ : x(k) \not \in B_ \epsilon(p)\}$ is finite.
\end{simplethm}

\begin{proof}
    ($\implies$) Suppose $\{ x(k) \}$ converges to $p$. Then we take an arbitrary $\epsilon > 0$. There exists a $K \in \mathbb{N}^+$ such that for $k \in \mathbb{N}^+$, $k \geq K$, $x(k)$ must belong to $B _ \epsilon(p)$. Then points $x(j)$ not belonging to $B_ \epsilon(p)$ must be of the form $x(j)$, where $j \in \mathbb{N}^+, j < k$. Since $j \in \{ 1, 2, \cdots, k - 1 \}$, we have finitely many such points.

($\impliedby$) Suppose that for any $\epsilon > 0$ there exists only finitely many indices such that their corresponding point is not in $B_ \epsilon (p)$. Let $S$ denote the aforementioned set of indices. In the case where this is the empty set, we can take $K = 1$. If the set is non-empty, then take $K$ as the largest such index plus one. 

    In both cases, all indices $k \geq K$ have $x(k) \in B_ \epsilon (p)$. We can prove this by contradiction: Suppose there exists some $j \geq K$, $x(j) \not \in B_ \epsilon(p)$. If $S = \emptyset$, then this is immediately a contradiction. Otherwise, then the existence of $j$ contradicts the maximality of $K$, since $K$ is defined as the largest such index plus one. 

As such a $j$ does not exist, we have that for all $k \in \mathbb{N}^+$, if $k \geq K$ then $x(k) \in B _ \epsilon (p)$.
\end{proof}

We cannot say that $\{ x(k) \}$ converges to $p$ is equivalent to for all $\epsilon > 0$, the set $\{ x(k) : k \in \mathbb{N}^+, x(k) \not \in B_ \epsilon(p) \}$ is finite. Consider the sequence $k \in \mathbb{N}^+$, $x(k) = (-1)^k$. This sequence has only finitely many points in its image, being $\{ 1, -1 \}$, but is clearly not convergent to any point $p \in \mathbb{R}$ (picking $\epsilon < \frac{1}{2}$ would work).
