\section{Day 6: Parametric, Implicit, and Explicit Form (Sep 15, 2025)}
Have $n < m$ be positive integers. 
\begin{definition}[Parametric Form]
    A set $S \subseteq \mathbb{R}^m$ can be written in parametric form (with $n$-variables) if there exists a set $A \subseteq \mathbb{R}^n$, and a continuous map $g : A \to \mathbb{R}^m$ such that 
    \[
    S = \{ g(x) : x \in A \}
    \]
\end{definition}

\begin{definition}[Higher Dimensional Graphs]
Let $f : A \subseteq \mathbb{R}^n \to \mathbb{R}^m$ be a continuous function. 
\[
    \text{graph}(f) = \{ (x, f(x)) \in \mathbb{R}^n \times \mathbb{R}^m \mid x \in A \}
\]
\end{definition}

In this course $\mathbb{R}^a \times \mathbb{R}^b = \mathbb{R}^{a + b}$, and we don't care whether the nesting of ordered pairs is done in the first or second entry.

\begin{definition}[Explicit Form]
    A set $S$ can be written in explicit form in $n$ variables if $S \subseteq \mathbb{R}^m$ is graph($f$) for some continuous function $f: A \subseteq \mathbb{R}^n \to \mathbb{R}^m$.
\end{definition}

\begin{definition}[Implicit Form]
    A set $S$ can be written in implicit form in $n$ variables if there exists a constant $c \in \mathbb{R}^m$, and a continuous function $f : A \subseteq \mathbb{R}^n \to \mathbb{R}^m$ such that $S = f^{-1}(\{ c \})$.
\end{definition}

Implicit form is a generalized version of a level set. The level set is defined for functions $g : \mathbb{R}^n \to \mathbb{R}$, notice that $g^{-1}(\{ c \}) = \{ x \in \mathbb{R}^n : g(x) = c \}$ is the same as the level set of $g$ at $c$.
