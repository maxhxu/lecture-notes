\section{Day 29: Implicit Function Theorem (Nov 17, 2025)}

We begin by stating some condition on the properties that a function that locally defines some variables in terms of the other ones must have.

\begin{simplethm}
Let $U \subseteq \mathbb{R}^n \times \mathbb{R}$ be open, $f : U \to \mathbb{R}$ be $C^1$. $(a, b) \in \mathbb{R}^n \times \mathbb{R}$, $a \in \mathbb{R}^n$, $b \in \mathbb{R}$. $f(a, b) = 0$, $f$ is not constant. If $f(x, y) = 0$ locally defines $y$ as a $C^1$ function $\phi: V \to W$ function of $x$ near $(a, b)$, for every point $v \in V$, $w = \phi(v)$, for every $j \in \{ 1, \cdots, n \}$,
    \[
    \frac{\partial f}{\partial x_j}(v, w) + \frac{\partial f}{\partial y}(v, w) \frac{\partial \phi}{\partial x_j}(v) = 0
    \]
\end{simplethm}

\begin{proof}
    Let $(x, y) \in \mathbb{R}^{n + 1}$. Let $\phi$ be the function that locally defines $y$ as a $C^1$ function of $x$ near $(a, b)$. Define $y = \phi(x)$, have $f(v, \phi(v)) = 0$. Since $f$ is also $C^1$, thus the chain rule applies to $h(x) := f(x, \phi(x))$. Computing,
   \begin{align*}
       \frac{\partial h}{\partial x_j}(v) &= \frac{\partial h}{\partial x_j}(v, w) \frac{\partial x_j}{\partial x_j}(v) +
       \left(\sum_{\substack{k = 1 \\ k \ne j}}^{n} \frac{\partial f}{\partial x_k}(v, w) \frac{\partial x_k}{\partial x_j}(v) \right) +
       \frac{\partial f}{\partial y}(v, w) \frac{\partial \phi}{\partial x_j}(v) \\
        &= \frac{\partial h}{\partial x_j}(v, w) \cdot 1 + \frac{\partial f}{\partial y}(v, w) \frac{\partial \phi}{\partial x_j}
    \end{align*}
    Notice that for $k \ne j$, $x_k$ is fixed in the computation of the partial w.r.t. $x_j$, so $\frac{\partial x_k}{\partial x_j}$ is 0 everywhere. Since $V$ is open, for all $v \in V$, $f(v, \phi(v)) = 0$ (constant on $V$), then $\frac{\partial h}{\partial x_j}(v) = 0$, which gives the desired result.
\end{proof}

\begin{theorem}[Implicit Function Thereom: Single Var]
    Let $U \subseteq \mathbb{R}^n \times \mathbb{R}$ be open, have $f : U \to \mathbb{R}$ be $C^1$. Let $(a, b) \in U$, $a \in \mathbb{R}^n$, $b \in \mathbb{R}$. \\
    If $f(a, b) = 0$, and $\frac{\partial f}{\partial y}(a, b) \ne 0$, then the equation $f(x, y) = 0$ locally defines $y$ as a $C^1$ function $\phi$ of $x$, with $(a, b) \in \text{dom}(\phi)$.
\end{theorem}

You don't know how big $\mathrm{dom}(\phi)$ is, and just like the inverse function theorem it is non-constructive. 

\begin{theorem}[Implicit Function Theorem: Multi Var]
    Let $U \subseteq \mathbb{R}^n \times \mathbb{R}^k$ be open. Let $F: U \to \mathbb{R}^k$ be $C^1$. Let $(a, b) \in U$ so $a \in \mathbb{R}^n$, $b \in \mathbb{R}^k$. If $F(a, b) = 0$, and the $k \times k$ matrix $\frac{\partial F}{\partial y}(a, b) = \left[ \frac{\partial F_i}{\partial y_j} \right]_{i, j}$ is invertible, then the equation $F(x, y) = 0$ locally defines $y$ as a $\mathbb{R}^k$ valued $C^1$ function $\phi$ of $x$ near $(a, b)$.
\end{theorem}

The implicit function informally says the following: Assume a solution exists to your nonlinear equation. If you can globally solve an approximate linear equation, then you can locally solve the nonlinear equation.


