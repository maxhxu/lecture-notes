\section{Day 5: Coordinate Transformations (Sept 11, 2025)}

\begin{definition}[Coordinate Transformation]
A \textbf{coordinate transformation} $f: A \to B$ is a continuous transformation that is usually bijective. $A$ and the map $f$ form a \textbf{coordinate system} for the codomain $B$.
\end{definition}

We want to plot subsets of $B$ and  describe them using the coordinate system defined by $f$ and $A$. We use $(u, v)$ to describe the elements in $A$, and $(x, y)$ for $B$, and would write
\[
g(u, v) = (u^2 + v^2, v)
\]
or simply
\[
(x, y) = (u^2 + v^2, v)
\]

\begin{definition}[Polar Coordinate Transformation]
A map $T: \mathbb{R}^2 \to \mathbb{R}^2$, with
\[
    T(r, \theta) = (r \cos \theta, r \sin \theta)
\]
\end{definition}
\noindent $T$ describes a map from polar coordinates to cartesian coordinates. The radius can be negative. Two notable properties are $T(r, \theta) = T(-r, \theta + \pi)$ and $T(r, \theta) = T(r, \theta + 2\pi)$, following from trigonometry.

The set $\{ (r, \theta) \in \mathbb{R}^2 : r = 2\}$ would describe the set $\{ (2 \cos \theta, 2 \sin \theta) \in \mathbb{R}^2 : \theta \in \mathbb{R} \}$, which corresponds to a circle of radius 2 centered at the origin. Restricting both sets to $\theta \in \mathbb{R}^+$ or $\theta \in \mathbb{R}^-$ would still correspond to the same set, meaning that the polar coordinate transformation is not injective. Another way to show is to consider the case $r = 0$.

\noindent What follows was not covered in lecture:
\begin{definition}[Cylindrical Coordinate Transformation]
A map $T: \mathbb{R}^3 \to \mathbb{R}^3$, with
\[
    T(r, \theta, z) = (r \cos \theta, r \sin \theta, z)
\]
\end{definition}
This is similar to the polar coordinate transformation except we add an additional $z$ field which remains unchanged. 
\begin{definition}[Spherical Coordinate Transformation]
A map $T: \mathbb{R}^3 \to \mathbb{R}^3$ with
\[
    T(\rho, \theta, \phi) = (\rho \cos \theta \sin \phi, \rho \sin \theta \sin \phi, \rho \cos \phi)
\]
\end{definition}
\noindent If this gets covered in class I'll make a writeup deriving the formula.

TODO: derive/motivate the formula
