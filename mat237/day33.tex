\section{Day 33: Second order partial derivatives and the Hessian (Nov 25, 2025)}

You can take partial derivatives of partial derivatives, written $\partial _i \partial _j f(a) := \partial_i (\partial _j f)(a) $. If $i \ne j$, we say that the partial is mixed. Otherwise, we say it is pure. We write $f_{xy} = (f_x)_y = \frac{\partial}{\partial y} (\frac{\partial f}{\partial x}) = \frac{\partial f}{\partial y \partial x}$, and that $\frac{\partial}{\partial x}(\frac{\partial f}{\partial x}) = \frac{\partial^2 f}{\partial x^2}$.

\begin{definition}[$C^2$]
    For open $U$, we say that $f : U \subseteq \mathbb{R}^n \to \mathbb{R}^m$ is $C^2$  if for $i, j \in \{ 1, \cdots, n \}$, $\partial_i \partial_j f$ exists and are continuous everywhere in $U$. 
\end{definition}

\noindent $f$ being $C^2$ means that $f$ is $C^1$, since each $\partial_j f$ are $C^1$ themselves on $U$. Being $C^1$ implies differentiability, and differentiability implies continuity. So all $C^2$ functions are also differentiable. 

\begin{simplethm}[$C^2$ Clairaut]
    If $f$ is $C^2$, then for $i, j \in \{ 1, \cdots, n \}$, $\partial_i \partial_j f = \partial_j \partial_i f$. 
\end{simplethm}

\noindent We say that the partials commute. There is a nice proof of this in the textbook, using the single variable MVT.

\begin{proof}

\end{proof}

\begin{definition}[Hessian]
    If a real valued $f$ is $C^2$ at some $a \in \mathbb{R}^n$. Define the Hessian of $f$ at $a$, a $n \times n$ matrix given by
    \[
    Hf(a) = [\partial_i \partial_j f(a)]_{i, j} = \begin{pmatrix}
        \partial_1 \partial_1 f(a) & \cdots & \partial_1 \partial_n f(a) \\
        \vdots & \ddots & \vdots \\
        \partial_n \partial_1 f(a) & \cdots & \partial_n \partial_n f(a) \\
    \end{pmatrix}
    \]
\end{definition}

\noindent For $\partial_i\partial_j f(a)$ to be a scalar, we need $f$ to be real-valued. 
