\section{Day 15: Derivatives of One Variable (Oct 06, 2025)}

\begin{remark}
    We did not learn differentiability until \ref{sec:differentiability}, but there are theorems that use this concept prior to its introduction. Also: 
``Professors say: `Never differentiate in public. You will only embarrass yourself.''' -Q
\end{remark}


\noindent Let $A \subseteq \mathbb{R}$ and let $f : A \to \mathbb{R}^m$ be a function. Let $a$ be an interior point of $A$. 
\begin{definition}[Differential]
    If $f$ is differentiable at $a$, define the linear map $d f_a: \mathbb{R} \to \mathbb{R}^m$ given by $d f_a(h) = f'(a)h$ called the differential of $f$ at $a$. 
\end{definition}

\noindent Vectors $v \in \mathbb{R}^m$ are identified with $m \times 1$  column vectors
\[
\begin{pmatrix}
v_1 \\ \vdots \\ v_m
\end{pmatrix}
\]
instead of $1 \times m$ row vectors, despite the fact that we write $v = (v_1, \cdots, v_m)$. This is because we can write for $v, w \in \mathbb{R}^m$, $v \cdot w = v^Tw = w^Tv$. We still distinguish between vectors and matrices.\\


\noindent From first year calculus, the linear approximation of $f: \mathbb{R} \to \mathbb{R}$ at a point $a$ is given by
\[
g(x) = f(a) + f'(a)(x - a)
\]
Substituting $x - a$ with $h$, we get that
\[
f(a + h) \approx f(a) + \underbrace{hf'(a)}_{df_a(h)}
\]
Rearranging, we get that $f(a+h) - f(a) - df_a(h) \approx 0$. In fact, we can strengthen this condition, such that
\[
\lim_{ h \to 0 } \frac{f(a+h) - f(a) - L(h)}{h} = 0
\]
where $L : \mathbb{R} \to \mathbb{R}^m$ is a linear transformation, such that $L(h) = f'(a)h = df_a(h)$. 

