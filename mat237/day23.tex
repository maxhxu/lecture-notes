\section{Day 23: Optimization (Nov 03, 2025)}

`Optimization is the best thing since sliced bread.' -Q

The generic optimization problem is of the following form:
\begin{problem}
Have $f$ be a real valued function on a set $S \subseteq \mathbb{R}^n$. \\
Find the global extrema of $f$ on $S$.
\end{problem}

Knowing the properties of $f$ and $S$ is useful. $f$ being differentiable/$C^1$, $S$ being bounded/compact, lets you apply our many theorems to help you know if extrema exist, and how to find them.

\begin{simplethm}[Optimization Lemma] \label{lem:optimization}
    Let $S \subseteq \mathbb{R}^n$, and $f : S \to \mathbb{R}$. If $S$ is compact, $f$ is continuous on $S$, and $f$ is differentiable on $S^o$, then the global extrema of $f$ occur on the boundary $\partial S$ or at a critical point $p \in S^o$, where $\nabla f(p) = 0$.
\end{simplethm}

\begin{proof}
    Since $S$ is compact, $S = S^o \cup \partial S$, with $S^o \cap \partial S = \emptyset$. By \ref{thm:evt}, we know there is a global extrema at some point $a \in S$. We apply \ref{thm:localevt}. Since $\nabla f$ exists at all points of $S$, $\nabla f(s) = 0$. Otherwise, $s$ must be on the boundary.  
\end{proof}

% Typically, the conditions of \ref{lem:optimization} are satisfied, and $S \subseteq \mathbb{R}^2$. Checking the points on the interior is easy. To check the boundary, if $S = [a, b] \times [c, d]$, then the boundary of $S$ consists of the lines $a$ to $b$, $b$ to $d$, $c$ to $d$, $a$ to $c$. Define a single variable function $g : [a, b] \to \mathbb{R}$ (or $[b, d], [c, d], [a, c]$). 

% If $S$ is the unit disk (2d circle) in $\mathbb{R}^2$, we can parameterize the boundary by $g : [0, 2\pi] \to \mathbb{R}$, $f(t) = (\cos t, \sin t)$.

% Once you have parameterized the boundary, use single variable optimization techniques on $f(g(x))$. You need to check the endpoints of the domain of $g$.

% The boundary can be more difficult to check in some cases, such as when it is unit sphere in $\mathbb{R}^3$. In this case, it is easier to consider $g : [0, 2\pi] \times [0, 2\pi] \to \mathbb{R}$, such that $g(\theta, \phi) = T(1, \theta, \phi)$ where $T$ is the spherical coordinate transformation, which hits every single point on $S^2$ and is also differentiable. Thus you only need to find global extrema of $f(g(\theta, \phi))$, with $\theta, \phi \in [0, 2\pi]$.
