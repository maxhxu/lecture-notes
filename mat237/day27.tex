\section{Day 27: Inverse Function Theorem (Nov 11, 2025)}

Diffeomorphisms are particularly nice, and there are lots of powerful theorems about diffeomorphisms. This motivates us to find ways to show that a function is a diffeomorphism.

\begin{simplethm} \label{thm:iftr}
    Let $U$ and $V$ be open subsets of $\mathbb{R}^n$, and $F : U \to V$ is a diffeomorphism. For every $x \in U$, $DF(x)$ is an invertible $n \times n$ matrix and the jacobian of the inverse $G : V \to U$ satisfies
    $DG(F(x)) = [DF(x)]^{-1}$
\end{simplethm}

\begin{proof}
    Have that for all $x \in U$, $G(F(x)) = Ix = x$. Since $F, G$ are diffeomorphisms, they are both $C^1$ on their domains, hence both differentiable on their domains by \ref{thm:c1impdiff}. The Jacobian of the identity is the identity itself.
    \begin{align*}
        D(G \circ F)(x) &= DG(F(x)) \circ DF(x) & \text{via chain rule} \\
        I &= DG(F(x)) \circ DF(x) & \text{jacobian of the identity}
    \end{align*}
    which proves the claim, since we found a (left) inverse for $DF(x)$, hence it must be invertible. 
\end{proof}

\begin{theorem}[Inverse Function Theorem] \label{thm:iftl}
    Let $A, B$ be open subsets of $\mathbb{R}^n$, with $a \in A$. Have $F: A \to B$ be $C^1$. If $DF(a)$ is invertible, then $F$ is a local diffeomorphism at $a$. 
\end{theorem}

\noindent Combining with \ref{thm:iftr}, we get that this is an if and only if relationship, meaning that a function is a local diffeomorphism at some point, if and only if it is $C^1$ and its Jacobian is invertible at said point. You would use this when computing a $C^1$ inverse seems extremely annoying/impossible. 
