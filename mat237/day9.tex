\section{Day 9: Open and Closed Sets (Sep 22, 2025)}

You should know that there exist sequential formulations equivalent to the definition of closed and open sets. We want sets that have nice properties: Let $A \subseteq \mathbb{R}^n$. We want a class of sets such that if a sequence in $\mathbb{R}^n$ converges to $a \in A$, then the \textit{tail} of the sequence belongs in $A$.

\begin{definition}[Open Set]
$A$ is open if every point of $A$ is an interior point of $A$.  
\end{definition}

This is equivalent to saying $A = A^o$, and $A \cap \partial A = \emptyset$.

\begin{definition}[Closed Set]
$A$ is closed if every limit point of $A$ belongs to $A$. 
\end{definition}

That is, $A^* \subseteq A$.

\begin{theorem}
    A set $A$ is open if and only if its complement $A^c = \mathbb{R}^n \setminus A$ is closed.
\end{theorem}

\begin{proof}
    We prove both directions:
    \begin{itemize}
        \item[$(\Rightarrow)$] Suppose $A$ is open. \\
        For contradiction, suppose that $A^c$ is not closed, meaning that there exists a limit point of $A^c$, being $p$, such that $p \not \in A^c$. Thus $p \in A$. But $p$ is an interior point of $A$, hence there exists an $\epsilon > 0$, $B_\epsilon(p) \subseteq A$. But $B_ \epsilon(p) \setminus \{ p \} \subseteq A$, this contradicts our assumption that $p$ is a limit point of $A$. 
        \item[$(\Leftarrow)$] Suppose $A^c$ is closed. \\
        Suppose that $A$ is not open. Then there exists a point $p \in A$, such that $p$ is not an interior point. Let $k \in \mathbb{N}^+$. Then for each such $k$ there exists a point $b_k \in B_ {\frac{1}{k}}(p)$, where $b_k \in A^c$. We check that this sequence $\{ b_k \}_k$ converges to $p$:
        \begin{itemize}
            \item[] Take $\epsilon > 0$. By the archimedean property, exist some $N \in \mathbb{N}$, with $\frac{1}{N} < \epsilon$. For all $b_j$, $j \geq N$, have $b_j \in B_ \frac{1}{j}(p)$.
        \end{itemize}
        But each $b_k \in A^c$. Since this sequence lies entirely in $A^c$, and $p \in A$, a limit point of $A^c$ doesn't belong to $A$. This contradicts $A^c$ being closed. 
    \end{itemize}
\end{proof}
