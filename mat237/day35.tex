\section{Day 35: Partitions (Jan 05, 2026)}

In single variable calculus, integrals are the area under a curve. $\int_a^b f(x) \, dx$ can be seen as the area between $0$ and $f$. We estimate by dividing up $[a, b]$ into partitions, where usually a finer partition (more pieces) yield a better estimate. In multi-variable calculus, we attempt to do the same.

\begin{definition}[Rectangle]
    A rectangle in $\mathbb{R}^n$ is a set $\mathbb{R} = [a_1, b_1,] \times \cdots \times [a_n, b_n]$, where for each $i$, $a_i < b_i$. The volume is given by $\prod_{i} (b_i - a_i)$.
\end{definition}

\begin{definitionbox}[Partition]
    An ordered $n$-tuple of sets $P = (P_1, \cdots, P_n)$ is a partition of $R$, if $P_j = \{ x_0^{(j)}, \cdots, x_{k_j}^{(j)} \}$ is a partition of the interval $[a_j, b_j]$ for each $j$ The index set of $P$ is the set 
\[
I = \{ (i_1, \cdots, i_n) \in \mathbb{N}^n : 1 \leq i_1 \leq k_1, \cdots, 1 \leq i_n \leq k_n \}
\]
    The $i$-subrectangle of $P$, with $i = (i_1, \cdots, i_n) \in I$ is given by 
\[ 
R_i = [x^{(1)}_{i_1 - 1}, x^{(1)}_{i_1}] \times \cdots \times [x^{(n)}_{i_n - 1}, x^{(n)}_{i_n}]
\]
\end{definitionbox}

The index set does not contain tuples containing 0.

\begin{simplethm}
    Let $P$ be a partition of some rectangle $R$, with $I$ being the index set of $P$ and $\{ R_i : i \in I \}$ be the subrectangles. 
    \begin{itemize}
        \item $\bigcup_{i \in I} R_i = R$
        \item For all $i, j \in I$, if $i \ne j$, $\Int(R_i) \cap \int(R_j) = \emptyset$
        \item $\sum_{i \in I} \Vol(R_i) = \Vol(R)$
    \end{itemize}
    
\end{simplethm}

