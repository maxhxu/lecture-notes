\section{Day 3: Graphs, Level sets, and Slices (Sept 8, 2025)}

Office hours starting next week! Check Quercus for more details.

Last class, we looked at functions $\mathbb{R} \to \mathbb{R}^n$. Today we look at functions $\mathbb{R}^n \to \mathbb{R}$. We develop graphs, level sets, and slices because they offer new ways to analyze and study properties of such functions, that cannot be easily captured otherwise. For example, it is difficult to visualize a more than 3 dimensional vector.

\noindent Have $A \subseteq \mathbb{R}^n$, $f : A \to \mathbb{R}^n$.

\begin{definition}[Graph]
    The graph of $f$ is $\{ (x, f(x)) : x \in A \}$ 
\end{definition}

Note that when you plot a function $f : A \subseteq \mathbb{R}^n \to \mathbb{R}$, the graph would exist in $\mathbb{R}^{n+1}$.

\begin{definition}[Level Set]
    The level set of $f$ at $k$ is $\{ x \in \mathbb{R}^n : f(x) = k \}$\footnote{If $A \subseteq \mathbb{R}^2$, the level set is called a \textit{contour}.}
\end{definition}

To produce the \textbf{slice} of a graph, we need to hold a coordinate $x_i$ constant, which we set to $a$. We call the following a $x_i$-slice, where the $c$ is at the $i$-th position in the tuple: 
\[
\{ (x_1, \cdots, x_{n+1}) \in \mathbb{R}^n : (x_1, \cdots, a, \cdots, x_n) \in A,  x_{n+1} = f(x_1, \cdots, a, \cdots, x_n)\}
\]
where in the first tuple, $x_i$ is omitted, with $x_i$ is replaced by $a$ in the second and third tuples. Note that the slice lives in $\mathbb{R}^n$, because we already have the information that $x_i = a$.

 In this course, you always want to specify the domain and codomain of your function, to avoid confusion. 

\begin{problem}
    Give a function $f: \mathbb{R}^2 \to \mathbb{R}$ whose level set at $0$ is the set 
    \[
    \{ (x, y) \in \mathbb{R}^2: |x| = |y| \}
    \]
\end{problem}

To solve these kinds of problems, you set $0 = |y| - |x|$ and you would get a candidate function $f(x, y) = |y| - |x|$. To show that two sets $A$ and $B$ are equal, you would typically prove that $A \subseteq B$, and $B \subseteq A$.

Alternatively you could define a function
\[
g(x, y) = \begin{cases}
    0 &\text{ if } |x| = |y| \\
    1 &\text{ otherwise}
\end{cases}
\]
which satisfies the requirements by construction.
