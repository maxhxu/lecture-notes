\section{Day 4: Vector Fields and Transformations (Sept 9, 2025)}

So far, we have seen parametric curves ($\mathbb{R} \to \mathbb{R}^n$), real valued functions ($\mathbb{R}^n \to \mathbb{R}$). Today we look at functions $\mathbb{R}^n \to \mathbb{R}^n$, which are called vector fields or transformations\footnote{very uncommon to call such functions transformations}.

\begin{definition}[Vector Field]
A $n$ dimensional vector field is a function $F: A \to B$ with $A, B \subseteq \mathbb{R}^n$.
\end{definition}

Note that vector fields are capitalized as per convention. 

(Not testable) Newton's law of gravity states that the force exerted by an object at the origin with mass $m_1$ on an object at $(x, y, z)$ with mass $m_2$ is given by
\[
F(x, y, z) = \frac{-Gm_1m_2}{||(x,y,z)||^2} \cdot \frac{(x, y, z)}{||(x, y, z)||}
\]

The magnitude is controlled by the first part of the product (note that there are only scalars), and the direction is controlled by the unit vector (note that it is scaled down to have a magnitude of 1).
