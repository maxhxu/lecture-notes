\section{Day 5: Autonomous and Exact ODEs (Jan 19, 2026)}

\begin{definition}[Autonomous]
    An ODE is autonomous if $y'$ doesn't depend on $t$.
\end{definition}

In other words, $y' = g(y)$. By setting $N(x) = 1$, $M(y) = \frac{1}{g(y)}$, we see that all autonomous ODEs are separable. Solving first order autonomous ODEs using the template for first order separable ODEs may miss the equilibrium (constant) solutions though.

In general, the equilibrium solutions of $y' = f(y)$ are $y = c$, where $f(c) = 0$.

\begin{definitionbox}
    \begin{itemize}
        \item The \textbf{critical points} of $y_0$ are values where $f(y_0) = 0$.
        \item $y_0$ is asymptotically stable for nearby values $y_1$, if the solutions of $y' = f(y)$, $y(0) = y_1$, have.
    \end{itemize}

\end{definitionbox}

% A trick is to leave the constants on the RHS.
